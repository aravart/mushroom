\documentclass[12pt]{beamer}
\usepackage{graphicx}
\usepackage{tikz}
\setbeameroption{hide notes}
\setbeamertemplate{note page}[plain]
\beamertemplatenavigationsymbolsempty
\hypersetup{pdfpagemode=UseNone} % don't show bookmarks on initial view
\definecolor{offwhite}{RGB}{249,242,215}
\definecolor{foreground}{RGB}{255,255,255}
\definecolor{background}{RGB}{24,24,24}
\definecolor{title}{RGB}{107,174,214}
\definecolor{gray}{RGB}{155,155,155}
\definecolor{subtitle}{RGB}{102,255,204}
\definecolor{hilight}{RGB}{102,255,204}
\definecolor{vhilight}{RGB}{255,111,207}
\definecolor{lolight}{RGB}{155,155,155}
\setbeamercolor{titlelike}{fg=title}
\setbeamercolor{subtitle}{fg=subtitle}
\setbeamercolor{institute}{fg=gray}
\setbeamercolor{normal text}{fg=foreground,bg=background}
\setbeamercolor{item}{fg=foreground} % color of bullets
\setbeamercolor{subitem}{fg=foreground}
\setbeamercolor{itemize/enumerate subbody}{fg=foreground}
\setbeamertemplate{itemize subitem}{{\textendash}}
\setbeamerfont{itemize/enumerate subbody}{size=\footnotesize}
\setbeamerfont{itemize/enumerate subitem}{size=\footnotesize}
\setbeamertemplate{footline}{\raisebox{5pt}{\makebox[\paperwidth]{\hfill\makebox[20pt]{\color{gray} \scriptsize\insertframenumber}}}\hspace*{5pt}}
\addtobeamertemplate{note page}{\setlength{\parskip}{12pt}}

\makeatletter
\def\beamer@framenotesbegin{% at beginning of slide
  \gdef\beamer@noteitems{}%
  \gdef\beamer@notes{{}}% used to be totally empty.
}
\makeatother

\AtBeginSection[]{
  \begin{frame}
  \vfill
  \centering
  \begin{beamercolorbox}[sep=8pt,center,shadow=true,rounded=true]{title}
    \usebeamerfont{title}\insertsectionhead\par%
  \end{beamercolorbox}
  \vfill
  \end{frame}
}

\newcommand{\State}{\ensuremath{\bm{\mathcal{S}}}}
\newcommand{\state}{\ensuremath{\bm{s}}}
\newcommand{\Model}{\ensuremath{\bm{\Theta}}}
\newcommand{\model}{\ensuremath{\bm{\theta}}}
\newcommand{\Sensitive}{\ensuremath{\bm{\mathcal{X}_S}}}
\newcommand{\Features}{\ensuremath{\bm{\mathcal{X}}}}
\newcommand{\features}{\ensuremath{\bm{x}}}
\newcommand{\Lab}{\ensuremath{\bm{\mathcal{Y}}}}
\newcommand{\lab}{\ensuremath{y}}
\newcommand{\Control}{\ensuremath{\bm{A}}}
\newcommand{\control}{\ensuremath{\bm{a}}}
\newcommand{\Policy}{\ensuremath{\bm{\Pi}}}
\newcommand{\policy}{\ensuremath{\bm{\pi}}}
\newcommand{\Hypothesis}{\ensuremath{\mathcal{H}}}
\newcommand{\discount}{\ensuremath{\gamma}}
\newcommand{\Disturbance}{\ensuremath{\bm{\mathcal{W}}}}
\newcommand{\disturbance}{\ensuremath{\bm{w}}}
\DeclareMathOperator*{\argmax}{argmax}
\DeclareMathOperator*{\argmin}{argmin}

\usepackage{amsmath,amsfonts,amsthm,bm,bbm}
\usepackage{algorithmic,algorithm}

\title{Title}
\author[Author]{Author}
\institute{Department of Computer Sciences, University of Wisconsin-Madison}

\begin{document}
\begin{frame}
  \titlepage
\end{frame}

\begin{frame}{}
  Let's say you have to add a new label to a text classifier, but oops you don't have any data.
\end{frame} 

\begin{frame}{}
  How can you use your previously collected labelled data and a few seed utterances to help you out?
\end{frame} 

\begin{frame}{}
  Modeling Assumptions: Let's say you could divide documents between context and keyphrase
\end{frame} 

\begin{frame}{}
  We can depict the corpus as a bipartite graph

  Introduce black edges...

  Introduce green edges...

  Introduce red edges...

  Introduce weights...

  Adding seeds to graph
\end{frame}

\begin{frame}{}
  This forms an electric network
\end{frame}
  
\begin{frame}{}
  The effective conductance provides a ranking
\end{frame}

\begin{frame}{}
  Practically speaking, it's hard to compute this over the whole graph, so we do some graph search to expose a subgraph
\end{frame}

\begin{frame}{}
  How can we refine this search with user feedback?
\end{frame}

\begin{frame}{}
  Idea: let the user revise the results and iterate
\end{frame}

\begin{frame}{}
  First round...
\end{frame}

\begin{frame}{}
  Second round...
\end{frame}

\begin{frame}{}
  We used electric networks, but we can plug in other things... (GPT-3... BERT...)
\end{frame}

\begin{frame}{}
  For the inner loop, we can do automated evalution by holding out one label from a dataset as a test set
\end{frame}

\begin{frame}{}
  For the outer loop, we can approximate the human's editing by matching items in a separate held-out set
\end{frame}

\begin{frame}{}
  For the outer loop, human experiments (Turkers) would be superior
\end{frame}

\begin{frame}{}
  BLEU score evaluation compares the synthesized corpus to the held-out corpus
\end{frame}

\begin{frame}{}
  But ultimately we want to see how a classifier performs over this synthesized
  corpus ...

  Stay tuned ...
\end{frame}

\end{document}

%%% Local Variables: 
%%% coding: utf-8
%%% mode: latex
%%% TeX-engine: xetex
%%% End: 