\documentclass[12pt]{beamer}
\usepackage{amsmath,amsfonts,amsthm,amssymb}
\usepackage{epsfig,graphicx}
\usepackage{mathrsfs}
\usepackage{soul,xcolor}
\usepackage[normalem]{ulem}
\usepackage{smartdiagram}
\usepackage{tikz}
\usepackage[siunitx]{circuitikz}
\usetikzlibrary{shapes,arrows}
\setbeameroption{hide notes}
\setbeamertemplate{note page}[plain]
\beamertemplatenavigationsymbolsempty
\hypersetup{pdfpagemode=UseNone} % don't show bookmarks on initial view
\definecolor{offwhite}{RGB}{249,242,215}
\definecolor{foreground}{RGB}{255,255,255}
\definecolor{background}{RGB}{24,24,24}
\definecolor{title}{RGB}{107,174,214}
\definecolor{gray}{RGB}{155,155,155}
\definecolor{subtitle}{RGB}{102,255,204}
\definecolor{hilight}{RGB}{102,255,204}
\definecolor{vhilight}{RGB}{255,111,207}
\definecolor{lolight}{RGB}{155,155,155}
\setbeamercolor{titlelike}{fg=title}
\setbeamercolor{subtitle}{fg=subtitle}
\setbeamercolor{institute}{fg=gray}
\setbeamercolor{normal text}{fg=foreground,bg=background}
\setbeamercolor{item}{fg=foreground} % color of bullets
\setbeamercolor{subitem}{fg=foreground}
\setbeamercolor{itemize/enumerate subbody}{fg=foreground}
\setbeamertemplate{itemize subitem}{{\textendash}}
\setbeamerfont{itemize/enumerate subbody}{size=\footnotesize}
\setbeamerfont{itemize/enumerate subitem}{size=\footnotesize}
\setbeamertemplate{footline}{\raisebox{5pt}{\makebox[\paperwidth]{\hfill\makebox[20pt]{\color{gray} \scriptsize\insertframenumber}}}\hspace*{5pt}}
\addtobeamertemplate{note page}{\setlength{\parskip}{12pt}}

\makeatletter
\def\beamer@framenotesbegin{% at beginning of slide
  \gdef\beamer@noteitems{}%
  \gdef\beamer@notes{{}}% used to be totally empty.
}
\makeatother

\AtBeginSection[]{
  \begin{frame}
  \vfill
  \centering
  \begin{beamercolorbox}[sep=8pt,center,shadow=true,rounded=true]{title}
    \usebeamerfont{title}\insertsectionhead\par%
  \end{beamercolorbox}
  \vfill
  \end{frame}
}

\newcommand{\State}{\ensuremath{\bm{\mathcal{S}}}}
\newcommand{\state}{\ensuremath{\bm{s}}}
\newcommand{\Model}{\ensuremath{\bm{\Theta}}}
\newcommand{\model}{\ensuremath{\bm{\theta}}}
\newcommand{\Sensitive}{\ensuremath{\bm{\mathcal{X}_S}}}
\newcommand{\Features}{\ensuremath{\bm{\mathcal{X}}}}
\newcommand{\features}{\ensuremath{\bm{x}}}
\newcommand{\Lab}{\ensuremath{\bm{\mathcal{Y}}}}
\newcommand{\lab}{\ensuremath{y}}
\newcommand{\Control}{\ensuremath{\bm{A}}}
\newcommand{\control}{\ensuremath{\bm{a}}}
\newcommand{\Policy}{\ensuremath{\bm{\Pi}}}
\newcommand{\policy}{\ensuremath{\bm{\pi}}}
\newcommand{\Hypothesis}{\ensuremath{\mathcal{H}}}
\newcommand{\discount}{\ensuremath{\gamma}}
\newcommand{\Disturbance}{\ensuremath{\bm{\mathcal{W}}}}
\newcommand{\disturbance}{\ensuremath{\bm{w}}}
\DeclareMathOperator*{\argmax}{argmax}
\DeclareMathOperator*{\argmin}{argmin}

\usepackage{amsmath,amsfonts,amsthm,bm,bbm}
\usepackage{algorithmic,algorithm}

\title{Instant Corpus -- Just Add Electricity}
\author[Author]{Ara Vartanian, Jerry Zhu}
\institute{Department of Computer Sciences, University of Wisconsin-Madison}

% Sometimes a data scientist is asked to train a text classifier for a new category without any training documents.  We propose one way for the data scientist to quickly generate a corpus for the new category.  Our method is based on the observation that many documents in an existing corpus -- which does not include the new category -- share structural similarity with the new category.  If we can find those documents and suggest modifications, we can generate promising candidate documents for the data scientist to use.  We do so with a graph-based approach, specifically by constructing an electric network and ranking candidate sentences by their effective conductance to a seed document.

\begin{document}

\setstcolor{red}

\begin{frame}
  \titlepage
\end{frame}

\begin{frame}{Prerequisites}
  \begin{itemize}
    \item An old corpus 
      \begin{itemize}
        \item Does not contain utterances of the new intent
        \item e.g.

          \texttt{I give this book five stars out of 6}

          \texttt{Can you add some disco to my playlist}

          \texttt{Show movies in the neighborhood}
        \end{itemize} 
      % \item A data engineer who can provide a seed utterance.
      % \begin{itemize}
      % \item e.g.

      %   \texttt{i would like to \underline{buy abbey road}}
      % \end{itemize}
  \end{itemize}
\end{frame}

  \begin{frame}{Workflow}
    \tikzstyle{io} = [trapezium, trapezium left angle=70, trapezium right angle=110, minimum width=3cm, minimum height=1cm, text centered, draw=black, fill=blue!30]
    \tikzstyle{process} = [rectangle, minimum width=3cm, minimum height=1cm, text centered, draw=black, fill=orange!30]
    \tikzstyle{decision} = [diamond, minimum width=3cm, minimum height=1cm, text centered, draw=black, fill=green!30]
    \tikzstyle{startstop} = [rectangle, rounded corners, minimum width=3cm, minimum height=1cm,text centered, draw=black, fill=red!30]
    \tikzstyle{line} = [draw, -latex']

\centering
    \resizebox{0.5\textwidth}{!}{
    \begin{tikzpicture}[node distance = 2cm, auto]
      \node (in1) [io] {Initialize (Step 1)};
      \node (in2) [startstop, below of=in1] {Seed};
      \node (pro1) [process, below of=in2] {Snowball (Step 2)};
      \node (cand) [startstop, below of=pro1] {Candidates};
      \node (dec1) [io, below of=cand] {Curate (Step 3)};
      \node (dec2) [startstop, below of=dec1] {Output};
      \path [line,-{stealth[length=3mm, width=2mm]}] (in1) -- (in2);
      \path [line,-{stealth[length=3mm, width=2mm]}] (in2) -- (pro1);
      \path [line,-{stealth[length=3mm, width=2mm]}] (pro1) -- (cand);
      \path [line,-{stealth[length=3mm, width=2mm]}] (cand) -- (dec1);
      \path [line,-{stealth[length=3mm, width=2mm]}] (dec1) -- (dec2);
      \path [line,-{stealth[length=3mm, width=2mm]}] (dec2) -| ([xshift=2cm, yshift=0cm]dec1.east) |- (in2);
    \end{tikzpicture}}
  \end{frame}
  
\begin{frame}{Step 1: Engineer provides seed utterance(s)}
\texttt{i would like to \underline{buy abbey road}}
\end{frame}
  
\begin{frame}{Step 2: Snowball generates ranked candidates.}

{\small \texttt{0.095 Where can I \underline{buy abbey road}}

\texttt{0.075 Where to \underline{buy abbey road}}

\texttt{0.060 I would like to \underline{buy abbey road}}

\texttt{0.057 I would like reservations to \underline{buy abbey road}}

\texttt{0.057 I would like you to \underline{buy abbey road}}

\texttt{0.057 I would like to add to \underline{buy abbey road}}

\texttt{0.056 me and katharine would like to \underline{buy abbey road}}

\texttt{0.055 i want to \underline{buy abbey road}}

\texttt{0.055 Would like to \underline{buy abbey road}}

\texttt{0.054 I have eight that would like to \underline{buy abbey road}}

\texttt{0.046 i need to \underline{buy abbey road}}

\bigskip 
etc (1809 results)}

\end{frame}

\begin{frame}{Step 3: Engineer curates the candidates}

  \texttt{Where can I \underline{buy abbey road}}

  \texttt{Where to \underline{buy abbey road}}

  \texttt{I would like to \underline{buy abbey road}}

  \texttt{I would like reservations to \underline{buy abbey road}}

  \texttt{I would like you to \underline{buy abbey road}}

  \texttt{I would like to add to \underline{buy abbey road}}

  \texttt{me and katharine would like to buy abbey road}

  \texttt{i want to \underline{buy abbey road}}

  \texttt{Would like to \underline{buy abbey road}}

  \texttt{I have eight that would like to \underline{buy abbey road}}

  \texttt{i need to \underline{buy abbey road}}

\end{frame}

\begin{frame}{Step 3: Engineer curates the candidates}

    \texttt{Where can I \underline{buy abbey road}}

    \texttt{Where to \underline{buy abbey road}}

    \texttt{I would like to \underline{buy abbey road}}

    {\color{red}\texttt{I would like reservations to \underline{buy abbey road}}}

    \texttt{I would like you to \underline{buy abbey road}}

    \texttt{I would like to add to \underline{buy abbey road}}

    \texttt{me and katharine would like to buy abbey road}

    \texttt{i want to \underline{buy abbey road}}

    \texttt{Would like to \underline{buy abbey road}}

    \texttt{I have eight that would like to \underline{buy abbey road}}

    \texttt{i need to \underline{buy abbey road}}

\end{frame}

\begin{frame}{Step 3: Engineer curates the candidates}

  \texttt{Where can I \underline{buy abbey road}}

  \texttt{Where to \underline{buy abbey road}}

  \texttt{I would like to \underline{buy abbey road}}

  \bigskip 

  \texttt{I would like you to \underline{buy abbey road}}

  \texttt{I would like to add to \underline{buy abbey road}}

  \texttt{me and katharine would like to buy abbey road}

  \texttt{i want to \underline{buy abbey road}}

  \texttt{Would like to \underline{buy abbey road}}

  \texttt{I have eight that would like to \underline{buy abbey road}}

  \texttt{i need to \underline{buy abbey road}}

\end{frame}

\begin{frame}{Step 3: Engineer curates the candidates}

  \texttt{Where can I \underline{buy abbey road}}

  \texttt{Where to \underline{buy abbey road}}

  \texttt{I would like to \underline{buy abbey road}}

  \bigskip 

  \texttt{I would like you to \underline{buy abbey road}}

  \texttt{I would like to add to \underline{buy abbey road}}

  \texttt{me and katharine would like to buy abbey road}

  \texttt{i want to \underline{buy abbey road}}

  \texttt{Would like to \underline{buy abbey road}}

  {\color{red}\texttt{I have eight that would like to buy abbey road}}

    \texttt{i need to \underline{buy abbey road}}

  \end{frame}

  \begin{frame}{Step 3: Engineer curates the candidates}

    \texttt{Where can I \underline{buy abbey road}}

    \texttt{Where to \underline{buy abbey road}}

    \texttt{I would like to \underline{buy abbey road}}

    \bigskip 

    \texttt{I would like you to \underline{buy abbey road}}

    \texttt{I would like to add to \underline{buy abbey road}}

    \texttt{me and katharine would like to buy abbey road}

    \texttt{i want to \underline{buy abbey road}}

    \texttt{Would like to \underline{buy abbey road}}

    {\small \texttt{I have decided that I would like to buy abbey road}}

    \texttt{i need to \underline{buy abbey road}}

  \end{frame}

\begin{frame}{Step 4: Get more results (Return to step 2)}

  
\texttt{Where can I find a copy of I \underline{buy abbey road}}

\texttt{Where and at what time can I \underline{buy abbey road}}

\texttt{Where and when can I see The I \underline{buy abbey road}}

\texttt{Can you find the I \underline{buy abbey road}}

\texttt{Can you find My I \underline{buy abbey road}}

\texttt{What animated movies can I \underline{buy abbey road}}

\texttt{When and where can I \underline{buy abbey road}}

\texttt{Where can I listen to the song The I \underline{buy abbey road}}

\texttt{Where can I see the trailer for Experienced I \underline{buy abbey
  road}}

\texttt{Where can I find the novel The Great I \underline{buy abbey road}}

\texttt{Can you search for Twilight I \underline{buy abbey road}}

\texttt{Can you play The Change I \underline{buy abbey road}}


etc

\end{frame}

  \begin{frame}{Workflow}
    \tikzstyle{io} = [trapezium, trapezium left angle=70, trapezium right angle=110, minimum width=3cm, minimum height=1cm, text centered, draw=black, fill=blue!30]
    \tikzstyle{process} = [rectangle, minimum width=3cm, minimum height=1cm, text centered, draw=black, fill=orange!30]
    \tikzstyle{decision} = [diamond, minimum width=3cm, minimum height=1cm, text centered, draw=black, fill=green!30]
    \tikzstyle{startstop} = [rectangle, rounded corners, minimum width=3cm, minimum height=1cm,text centered, draw=black, fill=red!30]
    \tikzstyle{line} = [draw, -latex']

\centering
    \resizebox{0.4\textwidth}{!}{
    \begin{tikzpicture}[node distance = 2cm, auto, scale=0.5]
      \node (in1) [io] {Initialize};
      \node (in2) [startstop, below of=in1] {Seed};
      \node (pro1) [process, below of=in2] {Snowball (+GPT3?)};
      \node (cand) [startstop, below of=pro1] {Candidates};
      \node (dec1) [io, below of=cand] {Curate};
      \node (dec2) [startstop, below of=dec1] {Output};
      \path [line,-{stealth[length=3mm, width=2mm]}] (in1) -- (in2);
      \path [line,-{stealth[length=3mm, width=2mm]}] (in2) -- (pro1);
      \path [line,-{stealth[length=3mm, width=2mm]}] (pro1) -- (cand);
      \path [line,-{stealth[length=3mm, width=2mm]}] (cand) -- (dec1);
      \path [line,-{stealth[length=3mm, width=2mm]}] (dec1) -- (dec2);
      \path [line,-{stealth[length=3mm, width=2mm]}] (dec2) -| ([xshift=5cm, yshift=0cm]dec1.east) |- (in2);
    \end{tikzpicture}}
  \end{frame}

\end{document}

%%% Local Variables: 
%%% coding: utf-8
%%% mode: latex
%%% TeX-engine: xetex
%%% End: 
